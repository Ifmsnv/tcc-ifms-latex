\chapter{Fundamentação teórica}

\section{PHP}
\lipsum[1]
\lipsum[2-3]

\section{Java}
\lipsum[1]

\subsection{Orientação a Objetos}

Programação Orientada a Objetos (também conhecida pela sua sigla POO) é um modelo de análise, projeto e programação de sistemas de software baseado na composição e interação entre diversas unidades de software chamadas de objetos.

\lipsum[2-3]

\subsubsection{Classes \& Objetos}

Em orientação a objetos, uma classe é uma descrição que abstrai um conjunto de objetos com características similares. Mais formalmente, é um conceito que encapsula abstrações de dados e procedimentos que descrevem o conteúdo e o comportamento de entidades do mundo real, representadas por objetos.[1] De outra forma, uma classe pode ser definida como uma descrição das propriedades ou estados possíveis de um conjunto de objetos, bem como os comportamentos ou ações aplicáveis a estes mesmos objetos.

Objeto é uma referência a um local da memória que possui um valor. Um objeto pode ser uma variável, função, ou estrutura de dados. Com a introdução da programação orientada a objetos, a palavra objeto refere-se a uma instância de uma classe.

Em programação orientada a objetos, um objeto passa a existir a partir de um ``molde'' (classe); a classe define o comportamento do objeto, usando atributos (propriedades) e métodos (ações).

\subsubsection{Herança}

Herança é um princípio de orientação a objetos, que permite que classes compartilhem atributos e métodos, através de ``heranças''. Ela é usada na intenção de reaproveitar código ou comportamento generalizado ou especializar operações ou atributos. O conceito de herança de várias classes é conhecido como herança múltipla. Sirelson Ramsey Como exemplo pode-se observar as classes ``aluno'' e ``professor'', onde ambas possuem atributos como nome, endereço e telefone. Nesse caso pode-se criar uma nova classe chamada por exemplo, ``pessoa'', que contenha as semelhanças entre as duas classes, fazendo com que aluno e professor herdem as características de pessoa, desta maneira pode-se dizer que aluno e professor são subclasses de pessoa. Também podemos dizer que uma classe pode ser abstrata(abstract) ou seja ela não pode ter uma instância, ela apenas ``empresta'' seus atributos e metódos como molde para novas classes.

\subsubsection{Polimorfismo}

Na programação orientada a objetos, o polimorfismo permite que referências de tipos de classes mais abstratas representem o comportamento das classes concretas que referenciam. Assim, é possível tratar vários tipos de maneira homogênea (através da interface do tipo mais abstrato). O termo polimorfismo é originário do grego e significa ``muitas formas'' (poli = muitas, morphos = formas).

O polimorfismo é caracterizado quando duas ou mais classes distintas têm métodos de mesmo nome, de forma que uma função possa utilizar um objeto de qualquer uma das classes polimórficas, sem necessidade de tratar de forma diferenciada conforme a classe do objeto.

Uma das formas de implementar o polimorfismo é através de uma classe abstrata, cujos métodos são declarados mas não são definidos, e através de classes que herdam os métodos desta classe abstrata.
